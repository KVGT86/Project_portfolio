\chapter{Sleeping}

Sleeping a MCU is important for embedded systems applications to
prolong battery life.


\section{Dynamic power consumption}

The dynamic power consumption for CMOS is
%
\begin{equation}
  P = f C V^2,
  \label{eqn:power}
\end{equation}
%
where $f$ is the clock frequency, $C$ is the total switched
capacitance, and $V$ is the power supply voltage.  It can be difficult
to reduce $V$ for a MCU and so the power consumption can only be
reduced by lowering the clock frequency, $f$, and/or shutting down
parts of the MCU to reduce $C$.


\section{Slowing the CPU clock}

The SAM4S, like many MCUs, can be clocked from a number of sources.
For example, it has a slow clock generated by an RC oscillator with a
frequency of 32768\,Hz.

With the mat91lib library the slow clock can be selected using:
%
\begin{minted}{C}
    mcu_select_slowclock ();
\end{minted}

\textbf{Warning}, switching to a slow CPU clock will cause havoc with
OpenOCD.  You might not be able to load a new program.  In this case,
you will need to \hyperref[erasing-flash-memory]{erase flash memory}.


\section{Disabling the CPU clock}

Further power reduction can be achieved by disabling the CPU clock.
In effect, this reduces $C$ in \refeqn{power}.  The clock is disabled
using the ARM WFI (wait for interrupt) instruction.  The clock remains
disabled until an interrupt occurs.  The WFI instruction is executed
by calling the mat91lib \code{cpu_wfi ()} function, defined as:
%
\begin{minted}{C}
static inline void
cpu_wfi (void)
{
    __asm__ ("\twfi");
}
\end{minted}

\textbf{Warning}, before executing the WFI instruction, it is necessary
to enable an interrupt to wake the CPU.

\textbf{Warning}, disabling the CPU clock will cause havoc with
OpenOCD.


\section{Disabling peripherals}

Further power reduction can be achieved by shutting down peripherals
that are not required.  This also reduces $C$ in \refeqn{power}.  The
drivers in mat91lib have a shutdown function, e.g., \code{spi_shutdown
  ()}.



\section{Current measurement}

Measuring the current to determine power consumption is not
straightforward.  Usually, the voltage drop across a known resistance
is measured.  For normal operation, this resistance needs a low value
otherwise there will be significant voltage drop and the MCU will not
run.  When sleeping, a much larger resistor is required so that the
voltage drop can be measured (all going well the MCU will only take a
few microamps when sleeping).  One approach is to use two voltage drop
resistors connected in parallel when the MCU is running normally, and
to switch out the low value one when the MCU is sleeping.  There are
special current measuring devices that dynamically vary the voltage
drop resistor.
